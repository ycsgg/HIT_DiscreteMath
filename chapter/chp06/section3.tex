\documentclass[../../main.tex]{subfiles}
\begin{document}

\section{路、圈、连通图}
\begin{enumerate}
    \item 如果是通道则不一定,因为如果存在通道 $v_1v_2\cdots v_n$ 则 $v_1v_2\cdots v_nv_{n-1}\cdots v_1v_2\cdots v_n$ 则是另一条通道。如果是迹则一定,假设有 $v_1v_2\cdots v_n$ 和 $v_1v^{'}_{2}\cdots v_n$ 两条迹,如果两条迹除去起始结束点没有重复点,则 $v_1v_2\cdots v_nv^{'}_{n-1}\cdots v^{'}_{2}v_1$ 是一个圈。否则假设 $v_i=v^{'}_{j}$ 是第一个重复的点则存在一个 $v_i$ 到 $v^{'}_{j}$ 的圈
    \item 对于 $p$ 归纳,$p=1,2$ 时显然成立,若 $p=k-1$ 时成立,则对于图 $(p,q)$ 删去任意顶点 $v$ 后有 $m$ 个分支 $G_1,G_2,\cdots G_m$。由归纳假设 $q_i \ge p_i-1$。则 $\sum\limits_{i=1}^{m}q_i=k-\deg v \ge k-1-m$ 由于图连通,所以 $v$ 到每个分支 $G_i$ 至少有一条边,即 $\deg v \ge m$ 从而 $q \ge k-1$
    \item 如果图是完全图则成立,否则假设有两个不相邻顶点 $u,v$ 有 $\deg u + \deg v \le p-2$ 删去这两个顶点,图 $G-\{u,v\}$ 的点数为 $p-2$,边数至少为 $\frac{(p-2)^2}{2}$ ,这是不可能的,所以对于任意不相邻顶点 $u,v$ 有 $\deg u + \deg v \ge p-1$,即图连通
    \item 证明思路同3
    \item 反证,假设存在两条不相交的最长路 $P_1,P_2$ ,任取 $v_1 \in P_1,v_k \in P_2$,由于图连通,则存在 $v_1v_2\cdots v_k$ 的一条路,取最大的 $l$ 使 $v_l \ in P_1$,最小的 $r$ 使 $r > l$ 且 $v_r \in P_2$ 则 $v_l \cdots v_r$ 中间的点都不在 $P_1$ 或 $P_2$ 中。通过这条路连接 $P_1,P_2$ 中较长的两部分,则得到的新路长度至少为 $|P_1|+1$ ,矛盾,得证
    \item 与 8 相同
    \item $\Rightarrow$ 任取 $v_1 \in V_1,v_k \in V_2$,有路 $v_1v_2\cdots v_k$,则存在 $v_{i}v_{i+1}$ 在路中且 $v_i \in V_1,v_{i+1} \in V_2$\\ $\Leftarrow$ $\forall v_1,v_n \in G$,取 $V_1=\{v_1\},V_2=V/\{v_1\}$,则存在 $v_2 \in V_2$ 使得 $v_1v_2 \in E$,之后取 $V_{1} = \{v_1,v_2\},V2=V/V_{1}$,如此进行,可知每次操作后均存在一条路从 $v_1$ 到 $v_i$,直到$v_i=v_k$
    \item 取图中最长路 $P=v_1v_2\cdots v_n$ 可知不存在 $v1v_k \in E$ 且 $v_k \notin P$ 否则 $v_k+P$ 是更长的路。则 $v_1$ 所有连边都是 $v_1v_i,v_i \in P$,由于$\deg v_1 \ge \delta(G)$,则至少存在一个 $i \ge \delta(G) + 1$ 使得 $v_1v_i$,则 $v_1v_2\cdots v_iv_1$ 是一个长度至少为 $\delta(G) + 1$ 的圈
    \item \begin{enumerate}
        \item 如果图 $G$ 不连通,则至少有一个支有 $q \ge p$ ,所以只需对连通图 $G$ 证明。如果 $\delta(G) > 1$ ,则由 8 可知存在长度至少为 $3$ 的圈,否则删去一个  $deg v = 1$ 的点,此时图 $G' = G - \{v\}$ 仍然满足 $q \ge p$,若还有 $\delta(G')=1$ ,则继续删去一个度数为 $1$ 的点,直到 $\delta(G') \ge 2$。因为点数为 $1,2$ 的图不可能有 $q \ge p$,而上述删点过程保证了 $q \ge p$ ,所以此时图点数至少为 $3$ 且 $\delta \ge 2$,至少存在一个大小为 $3$ 的圈
        \item 只需证明 $q=p+4$ 成立即可。 反证,假设存在一个图使得 $q = p+4$ 且不存在边不重的圈,取一个点数最小的这样的图记作 $G$,显然图中不存在度数为 $1$ 的点,否则删去这个点后图点数会更小同时仍然符合条件。且该图最小圈长度为 $5$,否则取一个长度小于等于 $4$ 的圈,删去这个圈上所有边后仍有 $q \ge p$,由上一问得到还有一个圈,不符合假设。下面证明 $\delta(G) \ge 3$,若存在 $\deg v = 2$,假设两条边为 $vv_1,vv_2$,可知$v_1v_2 \notin E$,否则有一个三元环。则删去 $v$ 加入边 $v_1v_2$,此时仍有 $q \ge p + 4$ 且满足性质,矛盾。故$\delta(G) \ge 3$,则 $2p+8=2q=\sum \deg v \ge 3p$ 即 $p \le 8$。此时取图中一个最小的的圈 $C$,$C$ 上的顶点除 $C$ 上的边外都有一条不在 $C$ 中的边,记 $S$ 为所有和 $C$ 上顶点距离为 $1$ 的点,则 $S$ 中点只有一条边与 $C$ 中顶点相连,否则会形成一个长度至多为 $\frac{|C|}{2}+2 < |C|$ 的环,矛盾。所以 $|S| > |C| \ge 5$ 且 $S$ 与 $C$ 顶点不重合,则 $p > |S| + |C| \ge 10$ 与 $p \ge 8$ 矛盾。综上不存在这样的图
    \end{enumerate}
    \item 取 $G$ 中最长路 $P=v_1v_2\cdots v_n$,可知 $v_1$ 所有边的终点在 $P$ 中,否则会有更长的路。对于 $v_1v_i$ 与 $v_1v_j$ 有 $|j-i| \ge 2$,否则存在长度为三的圈 $v_1v_iv_j$。$v_1$ 的 $k$ 条边依次为 $v_1v_2,v_1v_{i_2},v_1v_{i_3} \cdots v_1v_{i_k}$ 则 $i_k \ge 2+(k-1)\times2=2k$ 所以 $P$ 长度至少为 $2k$,则该图至少有 $2k$ 个顶点。同时让 $i_{j}=i_{j-1}+2$ 就得到 $2k$ 个顶点的图的唯一构造
\end{enumerate}
\end{document}